\documentclass[a4paper,10pt]{article}
\usepackage[utf8]{inputenc}
\usepackage{textcomp}
%opening
\title{Aussagenlogik --- Hr. Bartelt}
\author{Alexander Pietz}

\begin{document}

\maketitle

\begin{center}
\begin{tabular}[c]{ccccc}
\textbf{Deutsch} & \textbf{Latein} & \textbf{Englisch} & \textbf{Math.} & \textbf{Inf.}\\
Und & Konjunktion & AND & $\wedge$ & $ \& \& $\\
Oder & Disjunktion & OR & $\vee$ & $\mid\mid$\\
Nicht & Negation & NOT & $\overline{G}$ oder $\neg$ & $!$
\end{tabular}
\end{center}

\paragraph{Distrubutivgesetz}
  \[(A \wedge B) \vee C = (A \vee C) \wedge (B \vee C) \]
  \[(A \vee B) \wedge C = (A \wedge C) \vee (B \wedge C) \]

\paragraph{Axiome}
\begin{center}
\begin{tabular}{cc}
$A \wedge A = A$ & $A \vee A = A$\\
$A \wedge f = f$ & $A \vee f = A$\\
$A \wedge w = A$ & $A \vee w = w$\\
$A \wedge \overline{A} = f$ & $A \vee \overline{A}\overline{} = A$
\end{tabular}
\subparagraph{Regeln von De Morgan:}
\[\overline{A \vee B} = \overline{A} \wedge \overline{B}\]
\[\overline{A \wedge B} = \overline{A} \vee \overline{B}\]
\[\overline{\overline{A}} = A\]

\subparagraph{Satz des ausgeschlossenen Dritten:}
\[ (A \wedge B \wedge C) \vee (A \wedge B \wedge \overline{C}) = A \wedge B \]
\[ (A \vee B \vee C) \wedge (A \vee B \vee \overline{C}) = A \vee B \]
\end{center}


\end{document}
