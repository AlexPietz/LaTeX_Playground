\documentclass[a4paper,9pt]{article}
\usepackage[utf8]{inputenc}
\usepackage[T1]{fontenc}
\usepackage[ngerman]{babel}
\usepackage{amssymb}

\usepackage{fancyhdr}
\pagestyle{fancy}
\fancyhead{}
\fancyfoot{}
\fancyhead[C]{Charakterisierung Fabian $\bullet$ 29. Februar 2014 \\ R. Henke, R. Köhler, E. Kraus \& A. Pietz}
\fancyfoot[RO,LE]{\thepage}

%opening
\title{Fabian}
\author{René Henke, Robin Köhler, Eva-Maria Kraus \& Alexander Pietz}

\begin{document}

\maketitle

\section{Fabian... (allgemeine Charakterisierung)}
(keine Informationen über das Aussehen)
\begin{itemize}
 \item besitzt hauptächlich eine Beobachtungsrolle
 \item kommt aus der ``Arbeiterschaft''
 \item betrachtet und bewertet die Gesellschaft nach seinen Moralvorstellungen
 \item Fabians Moralismus geht aus Ereignisse in seinem Leben hervor (Siehe \ref{wp})
 \item[$\hookrightarrow$] ist ein perfekter Vertreter der neuen Sachlichkeit
 \item Werdegang, der für die Weimarer Republik nicht unüblich war
 \item ist Germanist
 \item ist ein idealistischer Mensch
 \item liebt seine Mutter, wie kaum jemand anderen
 \item hat kein festes Ziel vor Augen, während er durch die Stadt zieht
 \item scheint intelligent und gebildet
 \item hat eine Abneigung gegen die Kapitalgesellschaft 
 \item nutzt oft sehr zynischen Humor und gibt unkonventionelle Antworten (fühlt sich oft überlegen)
 \item lässt sich von der Gesellschaft mitziehen, macht hin und wieder Bemerkungen und zieht Schlussfolgerungen, die meist jedoch nicht an die Öffentlichkeit kommen
 \item ist grundlegend eher frustriert, glaubt jedoch trotzdem an das Gute
 \item lebt Ziel- und Rastlos in den Tag hinein
 \item strebt weder nach Geld noch nach Macht
 \end{itemize}

\section{Wendepunkte in Fabian's Leben}\label{wp}
\subsection{Die Arbeitslosigkeit}
\begin{itemize}
 \item Fabian verliert seine Arbeit als Werbetexter, einem (zumindest nach Meinung der Mutter) unangemessenen Job
 \item Fabian stellt sich durch diesen Zustand in Selbstironie
 \item Fabians Moralismus geht aus Ereignisse in seinem Leben 
 \item Die Arbeitslosigkeit könnte Auslöser für seine (passive) Kritik am ganzen System sein; Fabian möchte seinen Platz im System finden
 \item Durch Cornelia sucht Fabian sich nach keinen vorherigen Anstrengungen eine neue Arbeit
 \item $\rightarrow$ Die Jobsuche ist in Berlin erfolgslos, nach den weiteren Ereignissen verlässt er die Hauptstadt.
\end{itemize}

\subsection{Cornelia}
Cornelia entschwindet dem traditionellen Frauenbild komplett.Sie übernimmt viel Verantwortung für sich und verfolgt energisch ihre Karriere.
\begin{itemize}
 \item Fabian wird das erste mal selber aktiv und bleibt nicht Beobachter (z.B. der erste Kuss kam von ihm)
 \item Die traditionelle Männerrolle gefällt Fabian (Siehe als Kontrast Irene Moll)
 \item[$\hookrightarrow$] diese beinhaltet auch die Versorgung der Frau mit finanziellen Mitteln. Cornelia will sich Geld leihen, Fabian möchte darauf eingehen, ist aber aufgrund seiner Lage nicht dazu fähig.
 \item Cornelia verlässt Fabian zu Karrierezwecken
 \item nach der Trennung verliert er sein Selbstwertgefühl, weshalb er auch, um dieses wieder zu erlangen eine Nacht mit einer fremden Kunstreiterin verbringt
 \item[$\twoheadrightarrow$] Irene Moll und Cornelia Battenberg sind Fabian beide gesellschaflich überlegen. Einziger Trost nach dieser weiteren Niederlage ist die ``Flucht'' heim zu seiner Mutter
\end{itemize}


\subsection{Der Erfinder}
Der Erfinder hat eine neuartige Produktionsmaschine entwickelt, musste sich jedoch selber damit konfrontieren, dass daraufhin weniger Arbeitskräfte benötig, und tausende arbeitslos wurden
Er ist Gelehrter und wäre in wissenschaftlichen Kreisen hoch angesehenen.
\begin{itemize}
 \item Der Erfinder verkörpert einen „aktiven Fabian“, der nicht nur beobachtet und Schlüsse zieht, sondern auch handelt und sein Leben nach seinen Erkenntnissen ausrichtet
 \item Fabian sieht im Erfinder ein Idol
 \item Da Fabian bewundert, wie der Erfinder für die seines Erachtens richtige Sache kämpft, bietet er dem Erfinder an, bei ihm Unterschlupf zu finden, wenn er in Not gerät.
 \item[$\hookrightarrow$]Fabians Helfersyndrom wird deutlisch
 \item Es entsteht ein Vertrauensverhältnis $\mapsto$ Fabian stellt den Erfinder als seinen Onkel vor
\end{itemize}

\subsection{Tod Labudes}
Labude ist der Antagonist im Bezug auf Fabian. Er setzt sich für seine Moralvorstellungen ein, predigt diese und handelt eher aktiv als passiv
\begin{itemize}
 \item Weiterer Aspekt von Fabians Leben bricht weg
 \item Fabian wird aggressiv angesichts Labudes Tod, da dieser Sinnlos
 \item Ein der wenigen Stellen im Buch in dem man einen Einblick in Fabians Gefühlswelt bekommt
 \item Nach Labudes Tod hat Fabian - außer seiner Mutter - keinen anderen sozialen Kontakt mehr mit dem er sich regelmäßig trifft
\end{itemize}

\end{document}